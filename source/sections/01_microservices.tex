\section{Microservices - What and Why?}

\subsection{Einf�hrung}
kostenloses Buch unter: \\
\href{http://microservices-book.com/primer}{Download} \\
\href{C:/Users/annekathrin/Downloads/microservices-primer.pdf}{microservices-primer.pdf}
\\

\href{mailto:slideswjax2016workshop@ewolff.com}{slideswjax2016workshop@ewolff.com}
Antwort-Mail mit Slides \\

siehe: 01-MicroServices.pdf \\

\subsection{Definition}

\begin{itemize}
  \item independent deployment units (verschiedene unabh�ngige Einheiten, die
  sich einzeln bauen lassen)
  \item jede Technolgie m�glich
  \item jede Infrastruktur
  \item UI + Logik
\end{itemize}

\subsection{Vorteil} 
\begin{itemize}
  \item starke Entkopplung - einzelne Module
\end{itemize}


\subsection{Ziel} 
\begin{itemize}
  \item ein gro�es Gesamtsystem - m�ssen zusammen kolaborieren
\end{itemize}


\section{M�glichkeiten}
\subsection{Links}
\begin{itemize}
  \item Kolaborierung mittels Links
  \item Abstimmung von URLs
  \item Teams m�ssen wenig voneinander wissen
  \item gro�e Kopplung
\end{itemize}

\subsection{Datenreplikation}
bei Auswertungen -> Daten replizieren

getrennte Datenhaushalte

Rest und Messaging

\subsection{Unterteilung in Schichten}

am Beispiel E-Commerce-Shop

Frontends: iOS, Android, Web

Order
Search
Catalog
Customer
Rechnung

Search -- Datenreplikation --> Catalog
Order --> Customer
Order --> Billing

f�r Frontends -> Backends bauen --> zur Optimierung, angepasster Bestellservice

Idee:\\
wiederverwendbare Backendservices
Frontendspezifische Implementierungen

\subsubsection{Vorteile/ Benefits}
guter Weg zum bauen einer API
einfacher Weg zum "`reinmigrieren"'
�hnliche Struktur im Unternehmen -> Teams f�r iOS, Android, Web


\subsubsection{Nachteile}
Backends enthalten gleiche Logik f�r gleiche Prozesse -> Redundanzen
bei �nderungen sind mehrere Schichten betroffen -> Koordinierung, hoher
\textbf{Kommunikationsaufwand}
hoher Kommunikationsaufwand zwischen den Services
was ist wenn ein Service nicht antwortet?
mehr Komplexit�t


es macht nur Sinn, wenn die Komplexit�t meines Gesamtsystems sinkt
- Entkopplung von Teams


\subsection{Selft-contained Systems (SCS)}

scs-architecture.org

genau definert
- autonome Web.Anwendung
- optionale Service API

- Module enthalten auch den Webanteil (UI) - Daten \& Logik
-> Prozesse + UI �ndern

- keine geteilte UI und Business-Logik

- Aufteilung in fachliche Dinge + M�glichkeit zur Aufteilung aus fachlicher
Sicht

\subsubsection{einzelne Module}
Idee: 
Checkout-Modul
- Warenkorb kommt rein
- Umwandlung in Bestellung
- Anreicherung von Informationen

intern m�glich weiter aufzuteilen
Bsp. Ein-Klick-Prozess
normaler Bestellvorgang

Invoicing-Modul

Search-Modul

Browsing-Modul

\subsection{Micoroscervices vs. SOA}
% Tabelle
Kriterien				| Microservice | SOA
Deployment von Services | 1 Service = 1 deployment Einheit | 1 deployment
Einheit = n Services
Fazit | Entkopllung | schichtung

v�llig unterschiedlich


\section{Gr��e von Microservices - sollten klein sein}
\begin{itemize}
  \item m�glichst kleine Module - besser zu verstehen, maximal ein Team (7-8)
  \item einfacher zu ersetzen - Ersetzbarkeit
  \item Legacy Migration wird einfacher
  \item Umstieg auf neue Technologien einfacher (pro Service)
\end{itemize}

\section{Gr��e von Microservices - sollten gro� sein}

\begin{itemize}
  \item weniger Infrastruktur notwendig
  \item weniger Kommunikation notwendig
  \item gr��er und weniger - besser Umstrukturierung
\end{itemize}

\section{Conway's Law}
\begin{itemize}
  \item als Begrenzung sehen
  \item bzw. als enabler sehen
  \item technische und fachliche Trennung
\end{itemize}

\section{Microservice Bounded Context}
\subsection{Domain Driven Desgin}
\subsection{Context Relationships}
\begin{itemize}
  \item grundlegende Daten gemeinsam Nutzen
  \item gemeinsame Basis - wird von allen genutzt - starke Kopplung
  \item Customer/ Supplier - Team f�r Bestellprozess und Team f�r Registrierung
  \item Team f�r Bestellprozess gibt dem Registrierungsprozess vor, welche
  Daten n�tig sind
  \item conformist - nutzt die Daten die vorhanden, wenn �nderungen, sollte
  informiert werden - kann keine Vorgaben machen!
  
\end{itemize}

Registrierung 	-> Invoice (Datenfragmente von Registrierung vorhanden)
				-> Checkout (Datenfragmente von Registrierung vorhanden)
				
Sicherstellen, dass Daten in allen Modulen ankommen -> Event-Sourcing
Event-Sourcing: Events lassen sich speichern und immer wieder einspielen
Bsp. Kafka
oder ein zentraler Datenspeicher


Invoice					Checkout
BillingAdress
		||				||
		v				v
		Customer-Service		<- SharedKernel
		Name
		
Invoice muss immer beim Customer-Service anfragen


\section{Warum Microservces?}
\begin{itemize}
  \item (unabh�ngige) skalierbare Architektur
  \item Kommunikation begrenzen (Komponenten und Teams)
  \item einfacher Experimientieren
  \item ersetzbar
  \item strengere Modularisierung
  \item Robustheit- System ist stabiler, da nur ein Service ausf�llt und nicht
  das ganze System - Fehlermanagement - Umgang mit fehlerhaften Services
  \item weniger Risiko - Module ohne �nderungen nicht neu gebaut -> weniger
  Fehler
\end{itemize}

\section{Umgang mit Lagacy-Anwendungen}
\begin{itemize}
  \item Weggang vom Monolithen - komplexer Abh�ngigkeitsgraph
  \item nachhaltige Entwicklung
\end{itemize}

\subsection{(Deployment)-Monolith}
\begin{itemize}
  \item Architketur verrottet - schlecht strukturiert
  \item nicht mehr wartbar
  \item nicht ersetzbar
\end{itemize}

\subsection{Microservices}
\begin{itemize}
  \item Architekturverletzungen sofort sichtbar, da �ber "Rest"'-Schnittstelle
  nicht zugelassen
\end{itemize}

\section{Continuos Delivery}
\subsection{Probleme beim Monolithen}
\begin{itemize}
  \item komplexe Infrastruktur
  \item langsames Feedback
  \item gro�e Deployment-Enheit
  \item Deployment langsam
\end{itemize}

\subsection{Vorteile bei Microservices}
\begin{itemize}
  \item unabh�ngiges Deployment
  \item kleiner
  \item einfacher aufzubauen
\end{itemize}

Microservices nur mit Continuous Delivery!!!!!!

Integrations-Tests minimieren, da Flaschenhals
nur ein Microservices im Integrations\~test -> alle anderen m�ssen warten


Frage: Will ich Microservices machen?
		1. JA						2. NEIN
		/		\
asynchron		synchron
SCS				Layered



Standardisierung von 
\begin{itemize}
  \item Monitoring
  \item Logging
  \item Deployment/ config scripts
\end{itemize}

\subsection{Verteiltes System - Distributed System}
Regel: verteile deine Objekte NICHT!!